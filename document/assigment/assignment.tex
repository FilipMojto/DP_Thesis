\documentclass[12pt]{article}

\usepackage[slovak]{babel}

\begin{document}

\tableofcontents

\newpage

\section{Topic}


Zložitosť moderných softvérových produktov a kratší čas vývoja vedú k
potrebe zlepšiť kvalitu testovania. V tradičnom procese testovania sa
veľa času venuje hľadaniu chýb, najmä pri veľkých projektoch. Strojové
učenie môže výrazne zvýšiť efektivitu testovania predpovedaním možných
miest výskytu chýb. To umožňuje sústrediť zdroje na najzraniteľnejšie
oblasti systému, znížiť celkové náklady na testovanie a zvýšiť jeho
presnosť.

\newpage

\section{Assignment}

Analyzujte moderné metódy strojového učenia aplikovaných pri
testovaní softvéru. Konkrétne algoritmy či štruktúry rozdelťe
podľa známych techník aplikovaných v strojovom učení. Vyhodnotťe,
 ktoré metódy najviac spĺňajú praktický cieľ práce, a to preskúmaním
  už existujúcich riešení na predikciu sotvérových chýb identifikovaných
   z odbornej literatúry.

V rámci praktickej časti zhromaždite metadáta konkrétnych zdrojových kódov
 (históriu zmien kódu, záznamy o chybách či testovacie prípady). Na základe
  analýzy aplikujte odporúčanú techniku (prípadne ich kombináciu) pri navrhu
   a implementácii modelu strojového učenia. Postupujte pritom podľa známych
    princípov aplikovaných pri tvorbe modelu, od zberu a spracovania dát až
     po optimalizáciu a samotné nasadenie. Funkčnosť modelu overte na 
     zhromaždených údajoch.

Výsledky predikcie overte podľa metrík strojového učenia identifikova-ných 
v analýze. Model na záver porovnajte s už existujúcími modelmi a tradičnými 
metódami používaných pri testovaní softvéru. Vyhodnotťe rozdiely v rámci 
viacerých metrík (napr. presnosť alebo rýchlosť predikcie).

\newpage
\nocite{*}

\bibliographystyle{plain}
\bibliography{../literature/literature}

\end{document}